%%%%%%%%%%%%%%%%%%%%
% Copyright: Fernando Oleo Blanco.
%
% Thanks to all the people that helped my get to where I am.
%%%%%%%%%%%%%%%%%%%%

\documentclass[12pt, a4paper, twoside]{book} % General definition. In order to have the text completely centered, do not use the twoside option; however, twoside is recommended for printing.

% For general information and help, refer to https://www.overleaf.com/learn

%% PACKAGES

\usepackage[utf8]{inputenc} % Set the utf8 input encoding
\usepackage[english]{babel} % Select your prefered language. Example: \usepackage[activeacute,spanish]{babel}
\usepackage{amsmath, amsfonts, amssymb} % Mathematical notation
\usepackage{makeidx} % Index customization
\usepackage{fancyhdr} % Header and footer customization
\usepackage{titlesec} % Section printing customization
\usepackage{graphicx} % Graphics import and tools
\usepackage{svg} % To work with svg imports
\usepackage{geometry} % To modify the geometry of pages
\usepackage{booktabs} % To create beautiful tables
\usepackage{multicol} % To output text in two or more columns
\usepackage{hyperref} % Advance referencing
\usepackage{listings} % To add and format code
\usepackage[final]{pdfpages} % Includes pdfs directly, not as images
\usepackage{watermark} % Watermarks
\usepackage{lipsum} % Dummy text to test the design
\usepackage[backend=biber,
			style=numeric,
			citestyle=numeric,
			sorting=none]{biblatex} % To use biblatex, which is more complex, but as customizable as it gets. Please, modify this to your liking
			% IMPORTANT: this requires biber to be installed and run!!! If you are having issues with the bibliography, please, search for how to install and run biber!

%% PACKAGE CONFIGURATION

% Bibliography resource
\addbibresource{main.bib}

% Header and footer customization
\fancyhf{}
\fancyhead[LE]{\slshape \leftmark}
\fancyhead[RO]{\slshape \nouppercase \rightmark}
\fancyfoot[LE, RO]{\thepage}
% IMPORTANT! If your chapter/section names are too long to nicely fit on the header, use the shortened variant:
% \chapter[short tile]{actual long title, title} or \section[short title]{actual long title, title}

%% CUSTOM COMMANDS

% Add abstract to book style
\newenvironment{abstract}%
{\cleardoublepage \null \vfill \begin{center}%
		\bfseries \abstractname \end{center}}%
{\vfill\null}

%% GENERAL INFORMATION

\author{Your Name}
\title{The Title}
\date{\today}

%% DOCUMENT

% An example on how to add a watermark
{\thiswatermark{\centering\put(268,-802){\includegraphics[scale=1.5]{ComillasWatermark.pdf}}}}

\begin{document}
	
	\frontmatter
	
	% This creates a basic title page
	% It is recommended that if you want a more advance titlepage, that you define and create it completely from scratch.
	\begin{titlepage}
		\maketitle
	\end{titlepage}
	
	\begin{abstract}
		Abstract content
	\end{abstract}

	\clearpage\mbox{}\clearpage % This forces the next page to be odd-numbered, right hand page.

	% Thanks and other information page
	{\centering Thank yous\\}
	\vfill
	And other important information

	\clearpage\mbox{}\clearpage % This forces the next page to be odd-numbered, right hand page.
	
	\tableofcontents % Create the index
	\listoffigures % Create index of figures
	\listoftables % Create index of tables
	\lstlistoflistings % Create index of code sections
	
	\cleardoublepage % Open on right hand page (odd numbered)
	
	\mainmatter
	
	% Activate customized headers
	\pagestyle{fancy}
	
	\chapter{This is the first chapter}
\lipsum
\section{This is the first section}
\lipsum
\subsection{This is the first subsection}
\lipsum \cite{lovecraft2016el}.
 % This is the way we should import files and partition our document.
	
	% Include more stuff
	
	\appendix
	
	\chapter{This is the first appendix}
	
	\cleardoublepage
	\printbibliography[heading=bibintoc] % We print the entire bibliography of the document. The author recommends that, if the document contains tremendous amounts of references, that they should be written at the end of each chapter. Refer to https://www.overleaf.com/learn/latex/Bibliography_management_in_LaTeX for more information.
	
	\backmatter
	
\end{document}