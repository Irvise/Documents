\documentclass[12pt]{beamer}
\usepackage[utf8]{inputenc}
\usepackage[T1]{fontenc}
\usepackage{lmodern}
\usepackage[spanish]{babel}
\usepackage{amsmath}
\usepackage{amsfonts}
\usepackage{amssymb}
\usepackage{graphicx}
\usepackage{hyperref}
\usepackage{etoolbox} % to reduce spacing in bearmers toc
\usetheme[block=fill]{metropolis}

% to reduce spacing in bearmers toc
\makeatletter
\patchcmd{\beamer@sectionintoc}
{\vfill}
{\vskip\itemsep}
{}
{}
\makeatother

\begin{document}
	\author{Fernando Oleo Blanco \\ fernando.oleo@alu.comillas.edu \hfill 	\href{https://github.com/Irvise/Documents}{github.com/Irvise/Documents}}
	\title{Introducción a la Simulación}
	%\subtitle{}
	%\logo{}
	%\institute{ICAI - LinuxEC}
	\date{\today}
	%\subject{}
	%\setbeamercovered{transparent}
	\setbeamertemplate{navigation symbols}{}
	\setbeamertemplate{footline}[page number]
\begin{frame}[plain]
	\maketitle
\end{frame}

\begin{frame}{Información de la charla}
	\textbf{Duración estimada:} 2h \\
	\textbf{¡Haremos una simulación!} Así que por favor, celeridad
	
	\begin{block}{Esta charla no tratará}
		\begin{itemize}
			\item ANSYS \textregistered\ o ningún software en específico
			\item Métodos numéricos ni principios de convergencia
			\item Funcionamiento de los simuladores ni sus principios
			\item Análisis de sensibilidad ni optimizaciones de ningún tipo
			\item Mallado (meshing) de manera formal
			\item Diseño 3D, esto ya lo deberíais traer
			\item Caso práctico de post-procesado
		\end{itemize}
	\end{block}
\end{frame}

\begin{frame}{Índice}
	\setcounter{tocdepth}{2}
	\tableofcontents
\end{frame}

\section{Qué es la simulación y qué se requiere}

\begin{frame}{Requisitos mínimos para simular}
	\begin{block}{Entender muy, muy bien la física/ingeniería del problema}
		Los ordenadores no son tan inteligentes. Es un requisito indispensable entender muy bien lo aprendido en las clases. Dadle valor a la ingeniería.
	\end{block}
	Otros:
	\begin{itemize}
		\item Ganas. Los errores son constantes
		\item Leer la documentación. Hacer cursos/ver vídeos
		\item Práctica
	\end{itemize}
\end{frame}

\section{Software de simulación comercial}

\subsection{Software de pago}

\subsection{Software libre/gratuito}

\section{Los cinco pasos de toda simulación}

\begin{frame}{Las cinco fases para simular}
	\begin{itemize}
		\item Análisis del problema
		\item Creación de la geometría y grupos
		\item Mallado (meshing)
		\item Configuración de la simulación y simulación
		\item Post-procesado
	\end{itemize}
	\begin{block}{Todas son importantes}
		No os saltéis ninguna jamás. Especialmente el análisis y el post-procesado.
	\end{block}
\end{frame}

\section{Caso práctico}

\subsection{Presentación y análisis}

\subsection{Geometría y grupos}

\subsection{Mallado}

\subsection{Simulación}

\subsubsection{Fases generales de la simulación}

\subsubsection{Tipos de simulación}

\subsection{Post-procesado}

\begin{frame}{asd}
	content...
\end{frame}

\end{document}